\documentclass[11pt,pdf,aspectratio=129]{beamer}
\usepackage{bibentry}
\usepackage{graphicx} % Allows including images
\usepackage{booktabs} % Allows the use of \toprule, 
\usepackage{array}
\usepackage{wrapfig}
\usepackage{graphics}
\usepackage{graphicx}
\usepackage{amsfonts}
\usepackage{amssymb}
\usepackage{amsthm}
\usepackage{apacite}
\usepackage{textcomp}
% \usepackage{enumitem}
\usepackage{bibentry}
\usepackage{graphicx} % Allows including images
\usepackage{booktabs} % Allows the use of \toprule, 
% \usepackage[nottoc]{tocbibind}
\usepackage{threeparttable}



\title{Should we believe in HANK?}
\subtitle{Evidence from size-persistence tradeoff.}   
\author{\href{mailto://avlasov@nes.ru}{Vlasov Alexander}} 
\institute{NES x HSE}
\usetheme{Madrid}
% \usecolortheme{}

% \AtBeginSection{
% 	\begin{frame}
% 		\frametitle{Contents}
% 		\tableofcontents[currentsection]
% 	\end{frame}
% }

\begin{document}

\begin{frame}[fragile]
    \titlepage
\end{frame}


\section{Research question}

\begin{frame}\frametitle{What is HANK?}
\[\text{NK}= \text{New Keynesian}=\text{Monetary Policy is not Neutral}\]
\[\text{RANK}=\text{Representative Agent}+\text{NK}\]
\[\text{TANK}=\text{Two-Agent}\footnote{Sometimes referred as Spender-Saver Model}+\text{NK}=\text{One agent is Spender, one is Saver}+\text{NK}\]
\[\text{HANK}\footnote{The version by \citeA{KMV2018}}=\text{Heterogeneous Agent}+\text{{NK}}=\text{Heterogenity in saving portfolio}+\text{NK}
\]

\cite{Gali2018}

\end{frame}



\begin{frame}\frametitle{Outcomes of \citeA{KMV2018} model}
    \citeA{KMV2018} HANK model outcomes:
    \begin{enumerate}
        \item Size-Persistence trade-off: Cumulative elasticity of aggregate consumption declines with the increase in autocorrelation of monetary shock in a nonlinear manner.
        \item Inflation-Output Tradeoff: the same Taylor rule shocks lead to the increased effects in Inflation-Output tradeoff.
    \end{enumerate}


  
\end{frame}

\begin{frame}\frametitle{Picture of Size-Persistence trade-off}
    \begin{figure}\centering
        \includegraphics[scale=0.47]{SizePersistence.png}
        \caption{The difference between the New Keynesian models from \citeA{KMV2018} }
    \end{figure}
\end{frame}


\begin{frame}{Size-Persistent tradeoff by \citeA{KMV2018}, formally}
    \begin{align}
        \textit{RANK:} &\quad&\frac{d}{d\rho}\frac{-d\log C_0}{dR_0}&=0     \label{eq:SizePersistenceRANK}\\
    \textit{TANK:} &\quad&\frac{d}{d\rho}\frac{-d\log C_0}{dR_0}&\leq 0\footnotemark    \label{eq:SizePersistenceTANK}\\
    \textit{HANK:} &\quad& \frac{d}{d\rho}\frac{-d\log C_0}{dR_0}<0\quad\&\quad&
        \frac{d^2}{d\rho^2}\frac{-d\log C_0}{dR_0}<0
        \label{eq:SizePersistenceHANK}    
    \end{align}
    \footnotetext{Depending on the government adjustment.} 
    
\end{frame}


\begin{frame}\frametitle{Related literature}
\begin{block}{Main work}
    Model by \citeA{KMV2018}
\end{block}

\begin{block}{Microdata}
    \begin{itemize}
        \item     Working paper by \citeA{HolmPaulTischbirek2020}  find Inconsistent Evidence of HANK -- the response is larger than generated by HANK.
    \end{itemize}
\end{block}
\begin{block}{MPC}
    \begin{itemize}
        \item  Estimation of MPC's\footnote{Actually MPB, but they argue that it doesn't affect the results} by \citeA{Gross2020}:
        Increase of MPC is higher in 2008 than in 2011.
    \end{itemize}
\end{block}
% \begin{block}{Heterogenity in Portfolios}
%     \citeA{Luetticke2021} 
% \end{block}
\end{frame}


\section{Approach}

\begin{frame}{Empirical approach: Monetary Shock Identification I}
    Identification by \citeA[BRW]{BRW2021} two-step approach
\begin{enumerate}
    \item \citeA{FamaMacBeth1973} two-step estimation
    \item   \citeA{RigobonSack2004} heteroskedasticity estimator
\end{enumerate}
\end{frame}

\begin{frame}{Empirical approach: Monetary Shock Identification II}
    Assume that rate of interest: \begin{equation}
        \Delta R_{i,t}=\alpha_i+\beta_i e_t+\epsilon_{i,t},\label{eq:1}
    \end{equation}
    Let $e_t$ be normalized to have a one to one relationship with the 2-years Treasury yield.
    Than:
    \begin{equation}
        \Delta R_{i,t}=\theta_i+\beta_i \Delta R_{2,t}+\xi_{i,t},\label{eq:2}
    \end{equation}
Errors-in-variables!\footnote{If we estimate \eqref{eq:2} assuming the \eqref{eq:1} form.} It could be mitigated by \citeA{RigobonSack2004} heteroskedasticity estimator. Identification assumption: in dates of FOMC meetings the volatility is externally increased, compared to the week before.

When $\beta_i $ is estimated, the regression of rate on it to derive $e_t$.
\begin{equation}
    \Delta R_{i,t}=\alpha_i +e_t^{\textit{aligned}} \hat{\beta}_i+v_{i,t},
\end{equation}
\end{frame}


\begin{frame}{Empirical approach:  Main estimation}
    \begin{enumerate}
        \item Estimate elasticity of consumption to the deviation of interest rate from its natural value:
        \begin{align}
            \log \textit{Consumption}_{w}&=\alpha_w-\eta_w(r_{w}-r^{*}_w)+\varepsilon_w,\label{Consumption}\\
            (r_{w}-r^{*}_w)&=\alpha_w'+\gamma\sum_{0}^{\bar{w}} e^{\textit{aligned}}+\xi_w
        \end{align}
        \item Estimate the persistence of monetary shock: ARMA(p,q) model, first lag coefficient
        \item Regress elasticity of consumption to persistence of monetary shock:
        \begin{align}
            \eta&=\alpha_i+\beta_{1,i}L(\rho, i)+\xi_i,
            \label{eq:LinearLag}\\
            \eta&=\alpha_i'+\beta_{1,i}L(\rho, i)+\beta_{2,i}L(\rho,i)^2+\varepsilon_i,
            \label{eq:QuadraticLag}
        \end{align}
   
    \end{enumerate}
  \begin{block}{}
      If HANK is the model of choice,  we expect $\beta_{1,i}$ in \eqref{eq:LinearLag} to be zero, but  $\beta_{2,i}$ in \eqref{eq:QuadraticLag} to be significantly different from zero!
  \end{block}
\end{frame}






\section{Data}
\begin{frame}\frametitle{Data}
    Monetary Shock identification:
    \begin{itemize}
        \item  1-year, 2-year, 5-year, 7-year, 10-year, 20-year, and 30-year Treasury rates 
    \end{itemize}
Size-Persistence Trade-off:
\begin{itemize}
    \item Consumption as PCECC96 \footnote{Real Personal Consumption Expenditures.} \footnotemark[7]
        \item Inflation as a change in PCEPILFE\footnote{Personal Consumption Expenditures Excluding Food and Energy (Chain-Type Price Index).} 
        \item Natural (neutral) rate of interest by \citeA{HLW2017}\footnote{Cubic spline interpolation to monthly values.}
        \item 2-year Treasury rate as the Short-term rate ($r$).
    \end{itemize}
\end{frame}

% \begin{frame}{Summary Statistics}
    
% \end{frame}




\section{Results}

\begin{frame}{Results: Monetary Shock Identification I}
    \begin{table}[ht] \centering \tiny
        \begin{threeparttable}
        \caption{Monetary Shock Identification.  First step} 
        \label{tab:Betas} 
      \begin{tabular}{@{\extracolsep{1pt}}lcccccc} 
        \\[-1.8ex]\hline 
        \hline \\[-1.8ex] 
         & \multicolumn{6}{c}{\textit{Dependent variable:}} \\ 
        \cline{2-7} 
        \\[-1.8ex] & DGS1 & DGS5 & DGS7 & DGS10 & DGS20 & DGS30 \\ 
        \\[-1.8ex] & (1) & (2) & (3) & (4) & (5) & (6)\\ 
        \hline \\[-1.8ex] 
         DGS2 & 0.727$^{***}$ & 1.029$^{***}$ & 0.921$^{***}$ & 0.743$^{***}$ & 0.316$^{**}$ & 0.202 \\ 
          & (0.071) & (0.090) & (0.110) & (0.112) & (0.127) & (0.130) \\ 
          & & & & & & \\ 
         Constant & $-$0.005$^{***}$ & $-$0.001 & $-$0.0002 & 0.0002 & $-$0.001 & $-$0.001 \\ 
          & (0.001) & (0.002) & (0.002) & (0.002) & (0.003) & (0.003) \\ 
          & & & & & & \\ 
        \hline \\[-1.8ex] 
        Observations & 382 & 382 & 382 & 382 & 382 & 382 \\ 
        R$^{2}$ & 0.634 & 0.766 & 0.666 & 0.583 & 0.327 & 0.206 \\ 
        Adjusted R$^{2}$ & 0.633 & 0.765 & 0.665 & 0.582 & 0.325 & 0.204 \\ 
        Res. Std. Error & 0.028 & 0.035 & 0.043 & 0.044 & 0.049 & 0.051 \\ 
        Wald test & 103.9$^{***}$&129.9$^{***}$& 70.49$^{***}$&43.71$^{***}$& 6.201$^{**}$&2.406\\
        Wu-Hausman& 3.699$^{*}$& 0.002&0.259&0.847& 9.345 $^{***}$ & 8.707$^{***}$\\
        \hline 
        \hline \\[-1.8ex] 
      \end{tabular} 
      \begin{tablenotes}[flushleft]
    \item\tiny{}This table reports first stage of \citeA{BRW2021} monetary shock identification procedure for the FOMC announcement from 1994 to the most recent event 2021-04-28 (191 monetary events). OLS standard errors in the parenthesis. F-statistics on instrument insignificance is 44.030$^{***}$. Wu-Hausman stands for Hausman specification test for the endogeneity of a instrument $\left(\Delta R_{2,t}^{\textit{M}},-\Delta R_{2,t}^{\textit{NM}}\right)'$. $^{*}$p$<$0.1; $^{**}$p$<$0.05; $^{***}$p$<$0.01. 
      \end{tablenotes}
    \end{threeparttable}
      \end{table} 
      
\end{frame}

\begin{frame}\frametitle{Results: Monetary Shock Identification II}
    \begin{figure}[!ht]\centering
        \begin{minipage}{0.75\textwidth}
            \caption{Identified Monetary Shocks}
            \label{fig:BRW Shock}
            \includegraphics[width=\linewidth]{Identified monetary shock.pdf}
            {\begin{flushleft}\tiny The figure reposts \citeA[BRW]{BRW2021} method identified monetary shocks estimated from 1990-01-01 to the start of coronavirus related recession. US recessions are shaded. All shocks are significant with p-values less than 0.05\end{flushleft}}
            \end{minipage}
    \end{figure}


\end{frame}


\begin{frame}{Results: Elasticity of consumprion}
    
\begin{table}[!htbp] \centering \tiny
    \begin{threeparttable}
    \caption{Elasticity of consumption to $(r-r^*)$.} 
    \label{tab:TotalElasticityofConsumption} 
  \begin{tabular}{@{\extracolsep{5pt}}lcc} 
  \\[-1.8ex]\hline 
  \hline \\[-1.8ex] 
   & \multicolumn{2}{c}{\textit{Dependent variable: $\log\,$Consumption}} \\ 
  \cline{2-3} 
  \\[-1.8ex] & \textit{OLS} & \textit{IV} \\ 
  \\[-1.8ex] & (1) & (2)\\ 
  \hline \\[-1.8ex] 
   $(r-r^*)$ & 0.092$^{***}$ & 0.197$^{***}$ \\ 
   & (0.008) & (0.013) \\ 
   & & \\ 
  Constant & 9.095$^{***}$ & 9.050$^{***}$ \\ 
   & (0.011) & (0.014) \\ 
   & & \\ 
 \hline \\[-1.8ex] 
 Observations & 361 & 361 \\ 
 R$^{2}$ & 0.255 & $-$0.079 \\ 
 Adjusted R$^{2}$ & 0.253 & $-$0.082 \\ 
 Residual Std. Error  & 0.207 & 0.249 \\ 
 F Statistic & 122.922$^{***}$& \\
  Weak instrument& &508.1$^{***}$\\
  Wu-Hausman & &622.3$^{***}$\\
  \hline 
  \hline \\[-1.8ex] 
  \end{tabular} 
  \begin{tablenotes}[flushleft]
\item\tiny This table reports the results of estimation of  consumption elasticity to the deviation of rate from its neutral (natural) value, $(r-r^*)$.  Weak instrument stands for first stage F-statisitic, that indicate, whether the $\hat{R}$ is a strong instrument.
Wu-Hausman stands for Hausman specification test for the endogeneity of a instrument  $\hat{R}$.
$^{*}$p$<$0.1; $^{**}$p$<$0.05; $^{***}$p$<$0.01  
  \end{tablenotes}
\end{threeparttable}
  \end{table} 

\end{frame}

\begin{frame}\frametitle{Results: elasticity of consumption. In-windows estimation}

    \begin{figure}[ht]\centering
        \begin{minipage}{0.74\textwidth}
            \caption{Elasticity of Consumption to $(r_t-r_t^*)$. Window estimation}
            \label{fig:ElasticityofConsumption}
            \begin{minipage}{0.64\textwidth}
            \includegraphics[width=\linewidth]{consumption elasticity.pdf}
            \end{minipage}
            \begin{minipage}{0.35\textwidth}
            \includegraphics[width=\linewidth]{FirstStageElasticityPlot.pdf}
            \includegraphics[width=\linewidth]{WeakIV.pdf}
            \end{minipage}
            {\begin{flushleft}\tiny Consumption elasticity (panel A) is estimated from 1990-01-01 to the start of coronavirus associated recession with OLS and IV (instrumental variable: sum of monetary shocks). Weak instrument test  (Panel C) is the F-statistic of the instrument insignificance -- first stage F-test.  Estimated elasticity (and first stage regression coefficient) is significant with p-values less than 0.05. US recessions are shaded. \end{flushleft}}
            \end{minipage}
    \end{figure}
\end{frame}

\begin{frame}{Results: Monetary Shock Persistentce}
    

    \begin{figure}[!htbp]\centering
        \begin{minipage}{0.6\textwidth}
            \caption{Autocorrelation of Monetary Shocks}
            \label{fig:Autocorrelation}
            \includegraphics[width=\linewidth]{moneary_shocks_autocorr_plot.pdf}
            {\begin{flushleft}\tiny This figure presents the autocorrelation of monetary policy shocks estimated as the coefficient on the first lag in unrestricted ARMA$(p,q)$ estimation. Monetary shocks autocorrelation estimated from 1990-01-01 to the start of coronavirus associated recession. Estimated as coefficient on the first lag in ARMA estimated over \citeA{BRW2021} style identified monetary shock in a window of width equal to 31 month. US recessions are shaded.\end{flushleft}}
            \end{minipage}
    \end{figure}

\end{frame}


\begin{frame}\frametitle{Main Estimation Results: Linear Form}

    \begin{figure}[!htbp]\centering  
        \begin{minipage}{0.7\textwidth} \centering
            \caption{The effect of autocorrelation in monetary shock on the consumption-rate elasticity}
            \label{fig:MainRegressionCoefficients}
            \begin{minipage}{\textwidth}\centering
                \includegraphics[width=\linewidth]{Linear regression coefficient.pdf}
            \end{minipage}
        \end{minipage}
    \end{figure}
\end{frame}


\begin{frame}{Main Estimation Results: Quadratic Form}
    \begin{figure}
        \begin{minipage}{\textwidth}
        \includegraphics[width=0.5\linewidth]{Quadratic regression coefficient 1.pdf}
        \includegraphics[width=0.5\linewidth]{Quadratic regression coefficient 2.pdf}
        {\begin{flushleft}\tiny 
            The figure shows 3 coefficients $\beta$ in the regressions \eqref{eq:LinearLag} and \eqref{eq:QuadraticLag}.
            Standard errors are adjusted, heteroskedasticity and autocorrelation consistent. 
            The confidence intervals corresponds to  p$<0.1$, p$<0.05$, p$<0.01$.
    \end{flushleft}}
\end{minipage}
    \end{figure}

\end{frame}

\begin{frame}\frametitle{Results: Size-Persistence tradeoff. Table}
    \begin{table}
        \centering
    \begin{threeparttable}
        \caption{Regression coefficients on selected lags} 
     \label{tab:SelectedLags} 
\tiny
   \begin{tabular}{@{\extracolsep{1pt}}lccccccc} 
       \\[-1.8ex]\hline 
       \hline \\[-1.8ex] 
        & \multicolumn{6}{c}{\textit{Dependent variable: }Elasticity of consumption to $\left(r-r^*\right)$} \\ \\[-1.8ex] \cline{2-7} \\[-1.8ex]
        &   \multicolumn{2}{c}{$i=3$}&\multicolumn{2}{c}{$i=10$}&\multicolumn{2}{c}{$i=20$}&
       \\  \cline{2-3} \cline{4-5} \cline{6-7} 
       \\[-1.8ex] & (1) & (2) & (3) & (4) & (5) & (6)  \\ 
       \hline \\[-1.8ex]  
       $L(\rho, i)$ & $-$0.0708$^{***}$ & $-$0.0800 &   $-$0.1223$^{***}$ & 0.1458   &    $-$0.0924$^{**}$ & 0.2633$^{***}$   \\ 
       & (0.0350) & (0.1612) & (0.0458) &  (0.0980) & (0.0389)  &(0.0405)  \\ 
       & & & & & &   \\ 
       $L(\rho, i)^2$  &  & 0.11619 &   & $-$0.3639$^{**}$ &   &$-$0.4829$^{***}$   \\ 
       &  & (0.2176) &   &  (0.1363)&  &(0.0612)  \\ 
       & & & & & &  \\ 
      Constant & $-$0.0102$^{**}$ & $-$0.0102$^{**}$ & $-$0.0098 & $-$0.010& $-$0.0113 & $-$0.0117\\ 
       & (0.0049) & (0.0049) &(0.0069) & (0.0069) & (0.0091) & (0.0092) \\ 
       & & & & & &   \\ 
       \hline \\[-1.8ex] 
       $N$ & 345 & 345 & 338 & 338 & 328 & 328\\ 
       R$^{2}$ & 0.047 & 0.047 & 0.100 & 0.115 & 0.058 & 0.084  \\ 
       Adj R$^{2}$ & 0.044 & 0.041 & 0.109 & 0.130 & 0.055 & 0.079   \\ 
       Resid. SE & 0.050 & 0.050 & 0.049 & 0.048  & 0.051 & 0.050  \\ 
       F Statistic &  $4.096^{**}$      & $2.57^{*}$       & $7.115^{***}$       & $7.207^{***}$       & $5.63^{**}$      & $31.09^{***}$   \\
       \hline 
       \hline \\[-1.8ex] 
   \end{tabular} 
   \begin{tablenotes}[flushleft]
       \tiny\item This table reports results of linear estimations of the effect of autocorrelations of monetary shocks on the elasticity of consumption to the rate for some of the selected lags. 
       Heteroskedasticity and autocorrelation robust standard errors (Newey West, without prewhitening) in parenthesis.
     $^{*}$p$<$0.1; $^{**}$p$<$0.05; $^{***}$p$<$0.01.
     \end{tablenotes}
   \end{threeparttable}
\end{table}
\end{frame}

\begin{frame}{}
    \begin{figure}\centering
        \begin{minipage}{\textwidth}
            \caption{Comparing \citeA{KMV2018} with the result}
        \includegraphics[scale=0.2883]{SizePersistence.png}
        \includegraphics[scale=0.3]{DotPlot.pdf}
\begin{flushleft}\tiny  This plot  shows the joint distribution of consumption to excess rate elasticity with 15th lag of autocorrelation of monetary shock.
Standard errors are \emph{not} autocorrelation and heteroskedasticity adjusted, p$<$0.05.
           \end{flushleft}
    \end{minipage}
    \end{figure}
\end{frame}



\begin{frame}{Robustness I. Granger causality test}
    \begin{figure}[!htbp]\centering
        \begin{minipage}{0.55\textwidth}    
            \caption{Granger test for autocorrelation of monetary shock effect on consumption-deviation elasticity}
            \label{fig:GrangerTest}
            \vspace{1ex}
            \includegraphics[width=\linewidth]{GrangerTestPlot.pdf}
            \begin{flushleft}
                \tiny This figure 
            reports P-values for Granger F-test for significance of autocorrelation of monetary shock effect onto the consumption to deviation elasticity. 
            \end{flushleft}
        \end{minipage}
    \end{figure}


\end{frame}


\begin{frame}{Robustness II. Restricition to periods with strong instrument}
    \begin{figure}[!htb]\centering  
        \begin{minipage}{\textwidth} \centering
            \caption{The effect of autocorrelation in monetary shock on the consumption-rate elasticity estimated for periods, when $\hat{R}$ is strong instrument}
            \begin{minipage}{\textwidth}
                \includegraphics[width=0.32\linewidth]{Linear regression coefficient.Strong.pdf}
                \includegraphics[width=0.32\linewidth]{Quadratic regression coefficient 1.Strong.pdf}
                \includegraphics[width=0.32\linewidth]{Quadratic regression coefficient 2.Strong.pdf}
            \end{minipage}\\
            {\begin{flushleft}\tiny The coefficients $\beta$ in the regression \eqref{eq:LinearLag} and \eqref{eq:QuadraticLag}. Final sample is restricted to the area, where $\hat{R}$ is a strong instrument for elasticity of consumption estimation.
            Standard errors are adjusted, heteroskedasticity and autocorrelation consistent. The confidence intervals corresponds to  p$<0.1$, p$<0.05$, p$<0.01$.\end{flushleft}}
        \end{minipage}
    \end{figure}


\end{frame}

\begin{frame}{Robustness III. VAR estimation (Lag augmentation)}
    
\begin{figure}[!htbp]
    \begin{minipage}{\textwidth} \centering
        \caption{Impulse response function of VAR models corresponding to \eqref{eq:LinearLag}-\eqref{eq:QuadraticLag}.}
        \label{fig:IRF}
        \begin{minipage}{\textwidth}
            \includegraphics[width=0.32\linewidth]{IRF0.pdf}
            \includegraphics[width=0.32\linewidth]{IRF1.pdf}
            \includegraphics[width=0.32\linewidth]{IRF2.pdf}
        \end{minipage}
        \begin{flushleft}\tiny
       The confidence intervals are bootstrapped with 1000 runs. 
       The confidence intervals corresponds to  p$<0.1$, p$<0.05$, p$<0.01$, the darker the region, the less is p value. 
       Panel names stand for response function for impulse in stated variables in \eqref{eq:VAR1} and \eqref{eq:VAR2}. 
    \end{flushleft}
    \end{minipage}
\end{figure}
\end{frame}



\section{Conclusion}
\begin{frame}\frametitle{Conclusions}
    \begin{block}{So, should we believe in HANK?}

        The evidence above suggests that, we should. 
        At least we have found that consumption behaviour in size-persistent tradeoff corresponds to the HANK model.


    \end{block}
\end{frame}




\begin{frame}{Further work: Local Projection Estimation}
\begin{figure}[!htbp]
    \begin{minipage}{0.9\textwidth}
        \caption{Local Projection estimation}
        \begin{minipage}{0.6\textwidth}
            \includegraphics[width=\textwidth]{LPIVIRF_consumption_elasticity}
        \end{minipage}\begin{minipage}{0.39\textwidth}
    \includegraphics[width=\textwidth]{LPIVIRF_consumption_elasticity_OLS.pdf}
    \includegraphics[width=\textwidth]{LPIVIRF_consumption_elasticity_first_stage}
\end{minipage}
\begin{flushleft}\tiny
    This figure shows Local projections estimated with  AICc lag selection criterion,  maximum number of lags is 30.  Trend and quadratic trend included.
    Confidence intervals correspond to p$<$0.9.
\end{flushleft}
    \end{minipage}
\end{figure}
\end{frame}


\begin{frame}{Further work: unfinished part}
    Time Variation?


    Time-varying coefficient Local Projection model estimation!
    \begin{figure}[!h]
        \caption{Example by \citeA{Ruisi2019}}
        \includegraphics[width=0.6\textwidth]{Time-varying LP}
    \end{figure}
    Than we can estimate the \eqref{eq:LinearLag}-\eqref{eq:QuadraticLag} for each horizon of linear local projection.
\end{frame}


%Thanks
\begin{frame}

\begin{center}
    \Large Place for your suggestions and comments!
\end{center} 
\begin{center}
    \footnotesize
If you have any other suggestions/comments  please write \href{mailto://avlasov@nes.ru}{avlasov@nes.ru}
\end{center}


\end{frame}



\begin{frame}[t, allowframebreaks]
    \frametitle{References}
    \bibliographystyle{apacite}
    \bibliography{references}
    \end{frame}
\end{document}